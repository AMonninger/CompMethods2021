\input{./econtexRoot}\documentclass[pdflatex]{beamer}
\providecommand{\texname}{ProjectABM-Slides}% Indicate the keyname for the bibtex entry corresponding to this document
\providecommand{\texnameMaster}{ProjectABM}% Indicate the keyname for the bibtex entry corresponding to this document
\newif\ifdvi\dvifalse

%\usepackage{optional}
%\usepackage{ifthen}%\usepackage{\econtexRoot/BufferStockTeory}

% Can't read in ProjectABM.sty because some packages conflict with Beamer
% So need to redefine everything here

\usepackage{\econtexShortcuts}
\usepackage{natbib,amsmath,amssymb,rotating,subfigure}
\usepackage{verbatim,moreverb,graphicx}
\usepackage{wasysym}
\usepackage{dcolumn}
\usepackage{cancel}
%\providecommand{\LtxDir\EqDir}{\econtexRoot/Equations}
%\providecommand{\FigsRaw}{\econtexRoot/Code/Python/Figures}
%\providecommand{\CodeDir}{\econtexRoot/Code}
%\providecommand{\CalibrationDir}{\econtexRoot/Calibration}
\providecommand{\TableDir}{\econtexRoot/Tables}
%\providecommand{\ApndxDir}{\econtexRoot/Appendices}
%\providecommand{\Ex}{\mathbb{E}}

% additional packages: ABM
\usepackage{threeparttable}
\def\sym#1{\ifmmode^{#1}\else\(^{#1}\)\fi}

%\usepackage{natbib}\newcommand*{\newblock}{}

%\mode<presentation>
%{
	%  \usetheme{Warsaw}
	%  % or ...
	%  \setbeamercovered{transparent}
	%}

\mode<presentation>
{
	\usetheme{CambridgeUS}      % or try Darmstadt, Madrid, Warsaw, ...
	\usecolortheme{default} % or try albatross, beaver, crane, ...
	\usefonttheme{default}  % or try serif, structurebold, ...
	\setbeamertemplate{navigation symbols}{}
	\mode<beamer>{\setbeamertemplate{blocks}[rounded][shadow=true]}
	\setbeamertemplate{caption}[numbered]
	\useoutertheme{infolines}
	\useoutertheme[subsection=false]{miniframes}
} 

%\beamerdefaultoverlayspecification{<+->}

%\setbeamertemplate{navigation symbols}{}  % Take away navigation symbols

%\usetheme{Warsaw}

%\setbeamersize{text margin left=3mm}
%\setbeamersize{text margin right=3mm}

%_____________ Opening slide _______________________

%\title[Buffer Stock Theory]{Theoretical Foundations of Buffer Stock Saving}
%\author[Carroll]{Chris Carroll}
%\institute[JHU]{Johns Hopkins University}
%\date[\today]{September 12, 2019  \\ \medskip \medskip \medskip \href{https://econ-ark.org/}{\small Powered By} \\ \includegraphics[width=0.5in]{\econtexRoot/Resources/econ-ark-logo-small.png}}
\title[EPP, Expectations and Covid-19] {Equity Premium Puzzle, Expectations and Covid-19:\\ Evidence from Germany}
\subtitle{ \vspace{0.25em} \textit{-- Next Steps --} }
\author[A. Monninger]{Adrian Monninger\inst{1}}
\institute[]{\inst{1} Johns Hopkins University}
\date{July 11th, 2022}

\begin{document}
	\bibliographystyle{\econtexBibStyle}
	\begin{frame}
		\titlepage
	\end{frame}

\section{Main findings}
\begin{frame}
	\frametitle{My 2nd year paper showed that:}
	\begin{itemize}
		\item Lack of interest/ information very important, but not at heart of our models. Other \textit{softfactors as well}. I can measure the (relative) importance
		\item Active vs Passive vs Non-participants vs Non-adjuster. Especially active vs passive buyers has been overlooked. How stable is each type?
		\item Young, wealthy invested: Expectations and time important. Covid related?
	\end{itemize}
\end{frame}

\begin{frame}
	\frametitle{Next steps}
	2 Ways forward:
	\begin{enumerate}
		\item FIRE Model: Explicitly calibrating returns, participation, and adjustment costs
		\item Switcher Model: Switching between non-participants, active and passive buyers
%		\item Covid Model: Focusing on pandemic effects
	\end{enumerate}
\end{frame}

\section{FIRE Model}
\begin{frame}
	\frametitle{Agents in a FIRE model \textit{'know'} the following}
	\begin{itemize}
		\item Individual parameters:
		\begin{itemize}
			\item CRRA
			\item Discount Factor
			\item Income dynamics
		\end{itemize}
		\item Aggregate parameters:
		\begin{itemize}
			\item Return and Volatility
			\item Riskfree rate (alternative)
			\item Participation and adjustment costs (if added)
		\end{itemize}
		\item Complete history of returns, volatility, income realizations
		\item All choices: eg consumption and wealth paths	
		\end{itemize}
\end{frame}

\begin{frame}
	\frametitle{What have other models added? RETURNS}
	\begin{itemize}
		\item \cite{GSZ2008_TrustingStockMarketb}:\\
		Return = observed return + trust/distrust $\rightarrow$ Dispersion of returns by trust
		\item \cite{MN2011_DepressionBabiesMacroeconomic}: \\
		Return = function of age $\rightarrow$ Dispersion of returns by cohort
		\item Mateo:\\
		 Return = function of beliefs/education $\rightarrow$ Dispersion of returns by education and individuals
		\item \cite{ACGH2022_InformativeSocialInteractionsa}: \\
		Return = function of peers $\rightarrow$ Dispersion of returns by network
	\end{itemize}
\end{frame}

\begin{frame}
	\frametitle{What have other models added? Costs}
	\begin{itemize}
		\item \cite{V2002_ExplanationHouseholdPortfolio}: Per period stock market participation cost, fixed cost of trading, variable cost of trading stock
	\end{itemize}
	Additional factors:
	\begin{itemize}
		\item Information eg Awareness: \cite{GJ2005_AwarenessStockMarket}
		\item Financial literacy: \cite{vLA2011_FinancialLiteracyStocka}
		\item Participating peers: \cite{BISW2008_NeighborsMatterCausal}
		\item Not interested eg Disliking thinking about it: \cite{SB2012_MeasuringFinancialAnxietyb}
	\end{itemize}

\end{frame}

\begin{frame}
	\frametitle{What can I add?}
	\begin{itemize}
		\item Put them all together. Finding a functional form for returns, participation, and adjustment costs
		\item Return(beliefs, trust, peers, experience)
		\item Participation cost(Information, Literacy, Peers, Interest, Time)
		\item Adjustment costs(Peers, Time, Income)
		\item Calibrate total cost and value of each component
		\item Make counterfactual: aka let's shock something
	\end{itemize}
\end{frame}

\section{Switcher Model}
\begin{frame}
	\frametitle{What makes agents switch types?}
	\begin{itemize}
		\item Types: non-participants vs non-adjuster vs active vs passive buyer
		\item We know demographics and reasons for each group
		\item BUT: How stable is each group? How frequent to agents switch?
		\item Participation to non-participation: 20\% drop out in 2 year period (\cite{B2021_HouseholdStockMarket})
		\item 25\% Adjust in 2 year period (\cite{BCZ2012_ConsumptionSmoothingPortfolio})
	\end{itemize}
\end{frame}

\begin{frame}
	\frametitle{Data I would need for this}
	Problem with existing data: 
	\begin{itemize}
		\item Frequency: SCF is biannual
		\item Change in market value does not mean people adjusted the portfolio
	\end{itemize}
\vspace{2em}
	What do I need?
	\begin{itemize}
		\item High frequency data: Create a high frequency panel and follow each over a year?
		\item Retrospective: Ask agents when and why they bought. Eg retrieve a history by questionnaire
	\end{itemize}
\end{frame}

%\section{Covid Model}
%\begin{frame}
%	\frametitle{I have pandemic specific data, why not just use that?}
%	\begin{itemize}
%		\item Pandemic vs Crisis: Covid had different impact than a 'normal crisis'
%		\begin{itemize}
%			\item Time: people got bored
%			\item Return expectation: optimists (will last 3 weeks) vs pessimists (world will end)
%			\item Trust: no-none is to blame compared to GFC 
%			\item Social contact: Less office chat, but more crypto bros
%		\end{itemize} 
%	\end{itemize}
%\end{frame}

\section{Conclusion}
\begin{frame}
% Please add the following required packages to your document preamble:
% \usepackage{graphicx}
\begin{table}[]
	\resizebox{\textwidth}{!}{%
		\begin{tabular}{cllll}
			&  & \multicolumn{1}{c}{\textbf{Pros}} &  & \multicolumn{1}{c}{\textbf{Cons}} \\ \hline
			\multicolumn{1}{l}{} &  &  &  &  \\
			\textbf{\begin{tabular}[c]{@{}c@{}}FIRE \\ Model\end{tabular}} &  & \begin{tabular}[c]{@{}l@{}}+ Close to litereature\\ + Combining already existing models \\ makes start easier\\ + Comparison of factors\end{tabular} &  & \begin{tabular}[c]{@{}l@{}}- Limited data to actually calibrate\\ - Finding the 'right model' is hard\end{tabular} \\
			&  &  &  &  \\
			\textbf{\begin{tabular}[c]{@{}c@{}}Switcher \\ Model\end{tabular}} &  & \begin{tabular}[c]{@{}l@{}}+ Interesting gap in the litereature\\ + Cool model implications\\ + Could get help from Yujung Hwang\end{tabular} &  & \begin{tabular}[c]{@{}l@{}}- Very time intensive and costly to conduct\\ interview\\ - Retrospective is biased, but panel is costly\end{tabular} \\
%			&  &  &  &  \\
%			\textbf{\begin{tabular}[c]{@{}c@{}}Covid \\ Model\end{tabular}} &  & \begin{tabular}[c]{@{}l@{}}+ Unique data for that\\ + Low hanging fruit\end{tabular} &  & \begin{tabular}[c]{@{}l@{}}- Hard to convince people\\ - Can easily go wrong\end{tabular} \\
			&  &  &  & 
		\end{tabular}%
	}
\end{table}
\end{frame}
\bibliography{ProjectABM}
\end{document}\endinput