\begin{table}[]
	\caption{Summary Statistics: Reasons Bought}
	\label{tab:reason_bought}
	\resizebox{\textwidth}{!}{%
		\begin{threeparttable}
			\begin{tabular}{l|lccccccc}
				&  & \begin{tabular}[c]{@{}c@{}}Fully agree\\ (I)\end{tabular} &  & \begin{tabular}[c]{@{}c@{}}At least\\ rather agree\\ (II)\end{tabular} &  & \begin{tabular}[c]{@{}c@{}}Mean\\ (III)\end{tabular} &  & \begin{tabular}[c]{@{}c@{}}Standardized\\ (III)\end{tabular} \\ \hline
				&  &  &  &  &  &  &  &  \\
				low valuations &  & 38.74\% &  & 64.08\% &  & 2.79 &  & 0.90 \\
				plan &  & 43.54\% &  & 62.07\% &  & 2.76 &  & 0.92 \\
				time &  & 8.09\% &  & 26.59\% &  & 1.77 &  & -0.07 \\
				information &  & 7.60\% &  & 24.22\% &  & 1.70 &  & -0.15 \\
				less   consumption &  & 3.88\% &  & 18.73\% &  & 1.58 &  & -0.29 \\
				more income &  & 4.33\% &  & 19.88\% &  & 1.57 &  & -0.31 \\
				peer-effect &  & 4.15\% &  & 13.87\% &  & 1.49 &  & -0.36 \\
				bank fees &  & 0.38\% &  & 3.52\% &  & 1.21 &  & -0.65 \\ 
				&  &  &  &  &  &  &  &  \\\hline \hline
			\end{tabular}
			\begin{tablenotes}\footnotesize
				\item[] Summary statistics of reasons why households bought financial assets between March and August 2020. The first column reports the share of individuals who rated the reason 'fully agree', while the second column doe it for 'fully agree or 'rather agree'. The third column shows the mean (1-4 with 4 'fully agree') and the fourth column reports the mean of the standardized variable.
			\end{tablenotes}
		\end{threeparttable}
	}
\end{table}
