\begin{table}[ht!]
	\centering
	\caption{Summary Statistics: Reasons No Adjustment}
	\label{tab:reason_noadjust}
	\resizebox{\textwidth}{!}{%
		\begin{threeparttable}
			\begin{tabular}{l|lccccccc}
				&  & \begin{tabular}[c]{@{}c@{}}Fully agree\\ (I)\end{tabular} &  & \begin{tabular}[c]{@{}c@{}}At least\\ rather agree\\ (II)\end{tabular} &  & \begin{tabular}[c]{@{}c@{}}Mean\\ (III)\end{tabular} &  & \begin{tabular}[c]{@{}c@{}}Standardized\\ (III)\end{tabular} \\ \hline
				&  &  &  &  &  &  &  &  \\
				too risky &  & 20\% &  & 56\% &  & 2.5 &  & 0.3 \\
				high   valuations &  & 9\% &  & 49\% &  & 2.4 &  & 0.1 \\
				no time &  & 17\% &  & 49\% &  & 2.4 &  & 0.1 \\
				no savings &  & 18\% &  & 42\% &  & 2.3 &  & -0.1 \\
				peer-effect &  & 17\% &  & 36\% &  & 2.1 &  & -0.2 \\
				costs &  & 11\% &  & 32\% &  & 2.1 &  & -0.3 \\ 
				&  &  &  &  &  &  &  &  \\\hline \hline
			\end{tabular}
			\begin{tablenotes}\footnotesize
				\item[] Summary statistics of reasons why households did not adjust their portfolio between March and August 2020, but held stocks before. The first column reports the share of individuals who rated the reason 'fully agree', while the second column adds the answer 'rather agree'. The third column shows the mean (1-4 with 4 'fully agree') and the fourth column reports the mean of the standardized variable. The latter is constructed by using the average and standard deviation of all questions by each respondent.
			\end{tablenotes}
		\end{threeparttable}
	}
\end{table}