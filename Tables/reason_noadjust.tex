\begin{table}[]
	\caption{Summary Statistics: Reasons No Adjustment}
	\label{tab:reason_noadjust}
	\resizebox{\textwidth}{!}{%
		\begin{threeparttable}
			\begin{tabular}{l|lccccccc}
				&  & \begin{tabular}[c]{@{}c@{}}Fully agree\\ (I)\end{tabular} &  & \begin{tabular}[c]{@{}c@{}}At least\\ rather agree\\ (II)\end{tabular} &  & \begin{tabular}[c]{@{}c@{}}Mean\\ (III)\end{tabular} &  & \begin{tabular}[c]{@{}c@{}}Standardized\\ (III)\end{tabular} \\ \hline
				&  &  &  &  &  &  &  &  \\
				too risky &  & 20.47\% &  & 55.52\% &  & 2.53 &  & 0.31 \\
				high   valuations &  & 9.47\% &  & 48.62\% &  & 2.39 &  & 0.09 \\
				no time &  & 17.05\% &  & 49.39\% &  & 2.38 &  & 0.11 \\
				no savings &  & 18.25\% &  & 42.20\% &  & 2.30 &  & -0.06 \\
				peer-effect &  & 17.24\% &  & 36.07\% &  & 2.12 &  & -0.19 \\
				costs &  & 10.67\% &  & 32.40\% &  & 2.09 &  & -0.28 \\ 
				&  &  &  &  &  &  &  &  \\\hline \hline
			\end{tabular}
			\begin{tablenotes}\footnotesize
				\item[] Summary statistics of reasons why households did not adjust their portfolio between March and August 2020, but held stocks before. The first column reports the share of individuals who rated the reason 'fully agree', while the second column doe it for 'fully agree or 'rather agree'. The third column shows the mean (1-4 with 4 'fully agree') and the fourth column reports the mean of the standardized variable.
			\end{tablenotes}
		\end{threeparttable}
	}
\end{table}