\begin{table}[ht!]
	\centering
	\caption{Principal Component Analysis: Reasons No Adjustment}
	\label{tab:pca_reason_noadjust_alt}
	\resizebox{\textwidth}{!}{%
		\begin{threeparttable}
			\begin{tabular}{p{3cm}C{2cm}p{1cm}p{3cm}C{2cm}p{1cm}p{3cm}C{2cm}}%{lcclcclc}
				\hline
				\multicolumn{2}{c}{\begin{tabular}[c]{@{}l@{}}Comp   1\\ costs \end{tabular}} &  & \multicolumn{2}{c}{\begin{tabular}[c]{@{}l@{}}Comp   2\\ lack of savings\end{tabular}} &  & \multicolumn{2}{c}{\begin{tabular}[c]{@{}l@{}}Comp   3\\ time constraint\end{tabular}} \\ \cline{1-2} \cline{4-5} \cline{7-8} 
				&  &  &  &  &  &  &  \\
				too risky & 0.62 &  & no savings & 0.90 &  & no time & 0.99 \\
				costs & 0.60 &  & peer-effect & -0.37 &  &  &  \\
				peer-effect & 0.50 &  &  &  &  &  &  \\
				&  &  &  &  &  &  &  \\ \hline \hline
			\end{tabular}
			\begin{tablenotes}\footnotesize
				\item[] Principal component analysis of all factors from table \ref{tab:reason_noadjust_alt}. I use for each variable an indicator if the reason ranks above their own average and varimax rotation (no or promax rotation give similar results). Loadings above 0.32 are shown.
			\end{tablenotes}
		\end{threeparttable}
	}
\end{table}