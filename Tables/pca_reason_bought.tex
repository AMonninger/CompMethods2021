\begin{table}[ht!]
	\centering
	\caption{ Principal Component Analysis: Has Bought}
	\label{tab:PCA_reason_bought}
	\resizebox{\textwidth}{!}{%
		\begin{threeparttable}
			\begin{tabular}{llclcclc}
				\hline
				\multicolumn{2}{c}{\begin{tabular}[c]{@{}c@{}}Comp   1\\ additional resources\end{tabular}} &  & \multicolumn{2}{c}{\begin{tabular}[c]{@{}c@{}}Comp   2\\ active vs passive\end{tabular}} &  & \multicolumn{2}{c}{\begin{tabular}[c]{@{}c@{}}Comp   3\\ TBD?\end{tabular}} \\ \cline{1-2} \cline{4-5} \cline{7-8} 
				& \multicolumn{1}{c}{} &  &  &  &  &  &  \\
				costs & 0.57 &  & plan & \multicolumn{1}{l}{-0.69} &  & less consumption & \multicolumn{1}{l}{0.70} \\
				more income & 0.51 &  & low valuations & \multicolumn{1}{l}{0.58} &  & peer effect & \multicolumn{1}{l}{0.67} \\
				information & 0.49 &  &  &  &  &  &  \\
				time & 0.37 &  &  &  &  &  &  \\
				& \multicolumn{1}{c}{} &  &  &  &  &  &  \\ \hline
			\end{tabular}
			\begin{tablenotes}\footnotesize
				\item[] Principal component analysis of all factors from table \ref{tab:reason_bought}. I use for each variable an indicator if the reason ranks above their own average and varimax rotation (no or promax rotation give similar results). Loadings above 0.32 are shown.			\end{tablenotes}
		\end{threeparttable}
	}
\end{table}