\begin{table}[ht!]
	\centering
	\caption{Principal Component Analysis: Reasons No Participation}
	\label{tab:pca_reason_nopart}
	\resizebox{\textwidth}{!}{%
		\begin{threeparttable}
			\begin{tabular}{p{3cm}C{2cm}p{1cm}p{3cm}C{2cm}p{1cm}p{3cm}C{2cm}}%{lcclcclc}
				\hline
				\multicolumn{2}{c}{\begin{tabular}[c]{@{}c@{}}Comp   1\\ risk aversion\end{tabular}} &  & \multicolumn{2}{c}{\begin{tabular}[c]{@{}c@{}}Comp   2\\ lack of resources\end{tabular}} &  & \multicolumn{2}{c}{\begin{tabular}[c]{@{}c@{}}Comp   3\\ no savings\end{tabular}} \\ \cline{1-2} \cline{4-5} \cline{7-8} 
				&  &  &  &  &  &  &  \\
				too risky & 0.42 &  & no interest & 0.47 &  & no savings & 0.64 \\
				distrust & 0.42 &  & information & 0.40 &  & moral & -0.60 \\
				shock & 0.37 &  & no time & 0.40 &  &  &  \\
				high valuations & 0.35 &  & no savings & 0.34 &  &  &  \\
				&  &  & shock & -0.33 &  &  &  \\
				&  &  &  &  &  &  &  \\ \hline \hline
			\end{tabular}
			\begin{tablenotes}\footnotesize
				\item[] Principal component analysis of all factors from table \ref{tab:reason_nopart}. I use for each variable an indicator if the reason ranks above their own average and varimax rotation (no or promax rotation give similar results). Loadings above 0.32 are shown.
			\end{tablenotes}
		\end{threeparttable}
	}
\end{table}